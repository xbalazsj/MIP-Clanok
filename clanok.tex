% Metódy inžinierskej práce

\documentclass[10pt,twoside,slovak,a4paper]{article}

\usepackage[slovak]{babel}
%\usepackage[T1]{fontenc}
\usepackage[IL2]{fontenc} % lepšia sadzba písmena Ľ než v T1
\usepackage[utf8]{inputenc}
\usepackage{graphicx}
\usepackage{url} % príkaz \url na formátovanie URL
\usepackage{hyperref} % odkazy v texte budú aktívne (pri niektorých triedach dokumentov spôsobuje posun textu)

\usepackage{cite}
%\usepackage{times}

\pagestyle{headings}

\title{Minecraft\thanks{Semestrálny projekt v predmete Metódy inžinierskej práce, ak. rok 2022/23, vedenie: Meno Priezvisko}} % meno a priezvisko vyučujúceho na cvičeniach

\author{Ján Balázs\\[2pt]
	{\small Slovenská technická univerzita v Bratislave}\\
	{\small Fakulta informatiky a informačných technológií}\\
	{\small \texttt{xbalazsj@stuba.sk}}
	}

\date{\small 16. december 2002} % upravte



\begin{document}

\maketitle

\section{Abstrakt}

Keďže oblasť vzdelávacích videohier alebo serióznych hier nie je
obmedzená na hry, ktoré sú špeciálne navrhnuté na vzdelávacie účely,
videohry ako Minecraft vzbudili pozornosť učiteľov aj výskumníkov. Táto hra
má veľmi bohatú históriu bolo vydaných veľmi veľa verzíí. Povieme si taktiež
aj o tom že hra sa dá hrať aj na iných zariadeniach ako len na počítači.
Pri Minecrafte by som sa chcel hlavne priblížiť na jeho využitie v škole. Hra
ma veľmi bavila, a tak si myslím že jeho edukatívne využitie by bavilo aj
viacerých žiakov a študentov. Do hry sa dajú nainštalovať viac ako stotisíc
módov, ktoré pridávajú do hry nové itemy alebo vlastnosti.

\section{Úvod}

V článku sa predstavý survivalová hra Minecraft. V článku sa nachádzajú informácie čo vlastne minecraft je, jeho história, využitie v školstve, modifikácia hry, hra pre viacerých hráčov.



\section{O hre} 

Minecraft je sandboxová videohra. Hra môže byť pre viacerých hráčov alebo ju môžete hrať aj sám. V hre je niekoľko herných módov a to napríklad hra o prežitie kde si hráč buduje svet pomocou yískaných materiálov. Hráč musí získavať jedlo a prežiť noc kedy sa zjavujú príšery.

\section{Stavanie kocieck ako kreativita} 

Minecraft reinterpretuje rozprávanie o kreatívnych subjektoch a vynálezoch. Čítaním Minecraftu prostredníctvom dejín konštrukčných hračiek, moderných koncepcií kreativity a digitálnych praktík modifikácie tvrdím, že Minecraft umožňuje hráčom experimentovať s rôznymi sociálnymi a environmentálnymi podmienkami, ktoré sú historicky kľúčové pre debaty o formovaní kreatívnych subjektov, zakorenené v ostrovných naratívoch ako generickom mieste vyjednávania konfliktov medzi individuálnou autonómiou a sociálnou závislosťou. Ukazujem, ako sa hráči Minecraftu stávajú vynaliezavými subjektmi prostredníctvom simulácie Crusoeovho ostrova a vykonávania práce s kódom hry ako softvéru začleneného do sietí hráčov, programátorov a digitálnych technológií.

\section{Minecraft ako učebná pomôcka} \label{dolezita}
Táto štúdia skúma, ako môžu študenti - učitelia pracovať v malých skupinách pri riešení matematických úloh s využitím Minecraftu ako vzdelávacieho nástroja. Minecraft je komerčná hotová hra na báze kociek, často prirovnávaná k digitálnemu Legu, ktorá zahŕňa stavanie rôznych budov a figúrok. Je to hra s otvoreným svetom pre viacerých hráčov, vhodná na podporu tímovej práce a nelineárneho hrania. Patrí medzi najpredávanejšie hry všetkých čias a je obľúbená najmä medzi deťmi a mládežou. V tomto prípade sme využili Minecraft Education Edition, alternatívnu verziu hry vyvinutú špeciálne na použitie vo vzdelávacom prostredí. Táto verzia ponúka funkcie špeciálne navrhnuté pre pedagógov, ako napríklad jednoduché nastavenie servera, jednoduchý prístup študentov a ovládanie učiteľom s rôznymi možnosťami uľahčenia. Učitelia môžu kontrolovať, čo sa v hre deje, a môžu sa uistiť, že študenti sú v rovnakej komunite, napríklad nastavením a vypnutím násilia a rôznych znakov, ktoré môžu poškodiť herné avatary, vypnutím chatu alebo vypnutím možnosti, že herné postavy môžu byť poškodené pádom, ohňom a utopením. Minecraft sa čoraz častejšie používa vo vzdelávacom kontexte na celom svete. Niekoľko štúdií skúmalo jeho používanie vo vzdelávacích kontextoch. Iné štúdie skúmali, ako možno Minecraft využiť na vytváranie vedomostí žiakov v matematike, prírodných vedách, umení, jazykoch a spoločenských vedách. Predchádzajúce štúdie skúmali, ako môže Minecraft podporiť rozvoj priestorových schopností v oblasti prírodovedného, technického, inžinierskeho a matematického vzdelávania alebo informačnej gramotnosti. Napriek tomu je opodstatnený ďalší výskum týkajúci sa jeho dôsledkov na zlepšenie vyučovania v triede.


\section{Záver} \label{zaver} % prípadne iný variant názvu



%\acknowledgement{Ak niekomu chcete poďakovať\ldots}


% týmto sa generuje zoznam literatúry z obsahu súboru literatura.bib podľa toho, na čo sa v článku odkazujete
\bibliography{literatura}
https://www.sciencedirect.com/science/article/pii/S2666557322000222


https://www.scopus.com/record/display.uri?eid=2-s2.0-85007422229&origin=resultslist&sort=plf-f&src=s&st1=%22minecraft%22&st2=%22history%22&sid=6e1e8199050a701bade6d351940a310f&sot=b&sdt=b&sl=57&s=%28TITLE-ABS-KEY%28%22minecraft%22%29+AND+TITLE-ABS-KEY%28%22history%22%29%29&relpos=16&citeCnt=14&searchTerm=
\bibliographystyle{plain} % prípadne alpha, abbrv alebo hociktorý iný
\end{document}